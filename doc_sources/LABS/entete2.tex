%Partie essentielle
\documentclass[]{article}
\usepackage[T1]{fontenc}
\usepackage{minted}
\usepackage{fontspec}

\newcounter{chapterCounter}
\setcounter{chapterCounter}{2}%current chapter
\newcommand{\mytitle}{Chapter~\arabic{chapterCounter}: First steps with or-tools}
\title{\mytitle}
\author{Nikolaj van Omme \and Laurent Perron}

%%%%%%%%%%%%%%%%%%%%%%%%%%%%%%%%%%%%%%%%%%%%%%%%%%%%%%%%%%%%%%%%%%%%%%%%%%

%\usepackage[latin1]{inputenc}

%\usepackage{lmodern}
%\usepackage[usenames]{color}
\usepackage{geometry}
\geometry{hmargin=2cm,vmargin=2cm}
%fin 



%partie concernant la gestion des ent�tes
\usepackage{fancyhdr}
\pagestyle{fancy}
\usepackage{lastpage}
\renewcommand\headrulewidth{1pt}
\fancyhead[L]{Google or-tools lab sessions}
\fancyhead[R]{\mytitle}
\renewcommand\footrulewidth{1pt}
\fancyfoot[C]{\thepage/\pageref{LastPage}}
%\fancyfoot[R]{\today}
%fin





% Fancy Quotation Boxes in Latex
% from Jevon Wright

% for adjustwidth environment
\usepackage[strict]{changepage}

% for formal definitions
\usepackage{framed}

% environment derived from framed.sty: see leftbar environment definition
\definecolor{formalshade}{rgb}{0.95,0.95,1}
\definecolor{darkblue}{rgb}{0.1,0,0.55} 

\newenvironment{formal}{%
  \def\FrameCommand{%
    \hspace{1pt}%
    {\color{darkblue}\vrule width 2pt}%
    {\color{formalshade}\vrule width 4pt}%
    \colorbox{formalshade}%
  }%
  \MakeFramed{\advance\hsize-\width\FrameRestore}%
  \noindent\hspace{-4.55pt}% disable indenting first paragraph
  \begin{adjustwidth}{}{7pt}%
  \vspace{2pt}\vspace{2pt}%
}
{%
  \vspace{2pt}\end{adjustwidth}\endMakeFramed%
}

\newsavebox{\textBox}
\newlength{\textW}
\newlength{\insideW}
\newlength{\boxH}
\newlength{\boxW}
\newlength{\raiseLineHeight}
\newlength{\leftLineLength}

% usage \title_line{`Exercice`}{13}{1pt}{Difficulty}
\newcommand{\titleLine}[3]{%
%save text
\sbox{\textBox}{#1~#2}

%length of the text inside the box
\settowidth{\textW}{\usebox{\textBox}}

%length of the content inside the box, i.e. the length of the text + some margins
\setlength{\insideW}{\textW}
\addtolength{\insideW}{0.2cm}

%length and height of the framed text
\settoheight{\boxH}{\fbox{\parbox{\insideW}{\it \centering \usebox{\textBox}}}}
\settowidth{\boxW}{\fbox{\parbox{\insideW}{\it \centering \usebox{\textBox}}}}

%raise height
\setlength{\raiseLineHeight}{0.5\boxH}
\addtolength{\raiseLineHeight}{-\fboxsep}

%left line length
\setlength{\leftLineLength}{0.5\textwidth}
\addtolength{\leftLineLength}{-0.5\boxW}

%line with box
\noindent\raisebox{\raiseLineHeight}[\height][\depth]{\rule{\leftLineLength}{#3}}\fbox{\parbox{\insideW}{\it \centering \usebox{\textBox}}}\raisebox{\raiseLineHeight}[\height][\depth]{\rule{\leftLineLength}{#3}}\\
}


%Environment exercise

\newcounter{exerciseCounter}


\newenvironment{exercise}[1][1pt]%
{%
\stepcounter{exerciseCounter}
\titleLine{Mission}{\arabic{chapterCounter}.\arabic{exerciseCounter}}{#1}~\\
}%
{
%
}

%corps du document
\begin{document}
\maketitle
\thispagestyle{fancy} 
\section{What you will learn in this lab session}
We will first  test if your installation of the or-tools library is operational in section~\ref{working_settings}.
After being relieved it does work, we will write our first \emph{hello world} codes in section~\ref{hello_world}.

\section{Is my installed or-tools library operational?}\label{working_settings}

\section{Hello world}\label{hello_world}


\section{How to generate a criptaritymetic puzzle?}\label{how_to_generate_crytarithmetic_puzzle}
TOTO\\
\titleLine{Exercise}{1}{1pt}
\titleLine{Exercise}{11}{2pt}
\titleLine{Exercise de mes deux}{34}{3pt}
\titleLine{Debriefing Mission}{2.1}{0.5pt}
TIOTI\\

\begin{minted}[mathescape,
               linenos,
               numbersep=5pt,
               gobble=2,
               frame=lines,
               framesep=2mm]{c++}
  string title = "This is a Unicode \pi in the sky"
  /*
  Defined as $\pi=\lim_{n\to\infty}\frac{P_n}{d}$ where $P$ is the perimeter
  of an $n$-sided regular polygon circumscribing a
  circle of diameter $d$.
  */
  const double pi = 3.1415926535
\end{minted}

\begin{exercise}
 ceci est un exercice...
\end{exercise}


\begin{exercise}
 ceci en est un autre
\end{exercise}


\begin{exercise}[2pt]
 et un troisi�me plus compliqu�! et surtout avec beaucoup de textes le long de la premi�re ligne!
\end{exercise}

\begin{exercise}[4pt]
 et un 4e horriblement compliqu�!
\end{exercise}

\begin{formal}

Some text goes here.\\
and herer\\

what do you think?
\end{formal}

\newpage
Numerotation des pages.

\newpage
Numerotation des pages.

\newpage
Numerotation des pages.
\end{document}
%fin
