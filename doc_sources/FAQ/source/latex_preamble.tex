\usepackage{graphicx}
\usepackage{graphics}

% Meta data
\hypersetup {
    bookmarks=true,         % show bookmarks bar?
    unicode=false,          % non-Latin characters in Acrobats bookmarks
    pdftoolbar=true,        % show Acrobat's toolbar?
    pdfmenubar=true,        % show Acrobat's menu?
    pdffitwindow=false,     % window fit to page when opened
    pdfstartview={FitH},    % fits the width of the page to the window
    pdftitle={or-tools  user's manual},    % title
    pdfauthor={Google},     % author
    pdfsubject={User's manual for the Google or-tools library},   % subject of the document
    pdfcreator={Google},   % creator of the document
    pdfproducer={Google}, % producer of the document
    pdfkeywords={or-tools} {open source} {constraint programming} {operations research}, % list of keywords
    %pdfnewwindow=true,      % links in new window
    %colorlinks=true,       % false: boxed links; true: colored links
    %linkcolor=blue,          % color of internal links
    %citecolor=green,        % color of links to bibliography
    %filecolor=magenta,      % color of file links
    %urlcolor=cyan           % color of external links
}

%\definecolor{MyGray}{rgb}{0.80,0.80,0.80}

%\makeatletter\newenvironment{graybox}{%
%   \begin{lrbox}{\@tempboxa}\begin{minipage}{\columnwidth}}{\end{minipage}\end{lrbox}%
%   \colorbox{MyGray}{\usebox{\@tempboxa}}
%}\makeatother

%\makeatletter
%\renewenvironment{notice}[2]{
%  \begin{graybox}
%  \bf\it
%  \def\py@noticetype{#1}
%  \par\strong{#2}
%  \csname py@noticestart@#1\endcsname
%}
%{
%  \csname py@noticeend@\py@noticetype\endcsname
%  \end{graybox}
%}
%\makeatother

% To redefine footers and headers
\makeatletter
\fancypagestyle{normal}{
\fancyhf{}
\fancyfoot[LE,RO]{{\py@HeaderFamily\thepage}}
\fancyfoot[LO]{{\py@HeaderFamily\nouppercase{\rightmark}}}
\fancyfoot[RE]{{\py@HeaderFamily\nouppercase{\leftmark}}}
\fancyhead[LE,RO]{{\py@HeaderFamily \@title}} % here's the change
\renewcommand{\headrulewidth}{0.4pt}
\renewcommand{\footrulewidth}{0.4pt}
}
\makeatother

% Redefine these colors to something not white if you want to have colored
% background and border for code examples.
\definecolor{VerbatimColor}{rgb}{0.97,0.97,0.97} % light gray
\definecolor{VerbatimBorderColor}{rgb}{0,0,0} %black


% Redefine admonitions

\renewenvironment{notice}[2]%
  {\begin{Sbox}\begin{minipage}{0.10\textwidth}\includegraphics{dialog-warning.pdf}\end{minipage}\begin{minipage}{0.89\textwidth}}%
  {\end{minipage}\end{Sbox}\fbox{\TheSbox}}%

% To redifine the title and the disclaimer page
\renewcommand{\maketitle} {
\begin{titlepage}
  \thispagestyle{empty}
  \begin{center}
    \begin{flushright}
      \begin{minipage}[c]{0.5\textwidth}
	\vspace{2cm}
	\centering
	\mbox{\includegraphics[height=20mm]{google.pdf} \ \raisebox{+2mm}{\includegraphics[height=17mm]{logo.pdf}}}\\
	{\fontsize{18}{20}\selectfont or-tools open source library}
      \end{minipage}
    \end{flushright}

    \vspace{\stretch{0.5}}
  
    {\fontsize{50}{60}\selectfont User's Manual}\\
  \end{center}

\begin{flushright}
\begin{minipage}[c]{0.3\textwidth}
\large
\begin{flushright}
\vspace{2cm}

Nikolaj van Omme\\
%\vspace{0cm}
Laurent Perron  
\end{flushright}
\end{minipage}
\end{flushright}


  \vspace{\stretch{1}}
  \today \hspace{\stretch{1}}\copyright\ Copyright 2012, Google
\end{titlepage}

% Disclaimer page
\begin{titlepage}
{\Large Welcome to the or-tools user's manual!}\\

\vspace{\stretch{1}}

\copyright\ Copyright 2012, Google\\


\vspace{\stretch{1}}

\section*{How to reach us?}
The whole project \code{or-tools} is hosted on Google code (\url{http://code.google.com/p/or-tools/}). You can follow us on Google+ (\url{https://plus.google.com/u/0/108010024297451468877/posts}) and post 
your questions, suggestions, remarks, \ldots\ to the \code{or-tools} discussion group (\url{http://groups.google.com/group/or-tools-discuss}).

\section*{License information}
This document is provided under the terms of the 
\begin{center}
  \begin{large}\code{Apache License 2.0}\end{large}
\end{center}
You can find the complete license text at the following address: \url{http://www.apache.org/licenses/LICENSE-2.0}.\\

We kindly ask you not to make this document available on the Internet. This document should only be available at the following address:
\begin{center}
  \url{http://or-tools.googlecode.com/svn/trunk/documentation/documentation_hub.html}
\end{center}
This is the address of our documentation hub where you can find other useful sources of documentation about the \code{or-tools} library.

\section*{Trademarks}
GOOGLE is a trademark of Google Inc.\\
Linux is a registered trademark of Linus Torvald in the United States, other countries, or both.\\
Java and all Java-based trademarks and logos are trademarks of Sun Microsystem Inc. in the United States, or both.\\
Other companies, products, or service names may be trademarks or service marks of others.

\section*{Ackowledgments}
We thank the following people for their helpful comments:\\
Dania El-Khechen, Louis-Martin Rousseau

\section*{Accompanying code for this manual}

All the examples in this manual are coded in \code{C++}. You can find the complete examples on the documentation hub:\\

\begin{center}
  \url{http://or-tools.googlecode.com/svn/trunk/documentation/documentation_hub.html}
\end{center}

or under the directory:\\

\code{documentation/tutorials/C++}\\

of the or-tools library.\\

If you prefer to code in \code{Python}, \code{Java} or \code{C\#}, we have translated all the examples in your favorite language. You can find the complete examples on the documentation hub or under the directories:\\

\code{documentation/tutorials/Python}\\
\code{documentation/tutorials/Java}\\
\code{documentation/tutorials/Csharp}.\\

~\\
{\Large Happy reading!}
\end{titlepage}
}

